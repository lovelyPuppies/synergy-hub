\documentclass[12pt]{article}
\usepackage{emoji}
\usepackage{fontspec} % Required for font settings
\usepackage[utf8]{inputenc}

% Set the main font to Noto Sans CJK KR for Korean and English
\setmainfont{Noto Sans CJK KR}
\newfontfamily\emojiFont{Noto Color Emoji} % Optional: for explicit emoji handling

\usepackage{xcolor}   % Required for color settings
% Set background color and text color
\pagecolor{black}  % Set the background color to black
\color{white}      % Set the text color to white

\usepackage{amsmath}  % Required for mathematical symbols
\usepackage{graphicx} % Required for including images
\usepackage{tikz}     % 3D 그래프 그리기
\usepackage{pgfplots} % 3D 그래프 추가
\usepackage{hyperref} % 하이퍼링크 추가

% Set up hyperref options
\hypersetup{
    colorlinks=true,     % 하이퍼링크에 색상을 적용
    linkcolor=cyan,      % 목차와 섹션 링크의 색상
    urlcolor=cyan,       % URL 링크 색상
    pdftitle={외적 (Cross Product)},
    pdfauthor={Author: wbfw109v2},
    pdfsubject={벡터 간 관계의 수학적 모델과 응용}
}

\title{
    외적 (Cross Product): \\
    벡터 간 관계의 수학적 모델과 응용
}
\author{Author: wbfw109v2}
\date{\today}

\begin{document}

% Title and Table of Contents
\maketitle
\tableofcontents

\section{배경과 역사}

\noindent 벡터의 외적(Cross product)은 두 벡터 사이의 기하학적 관계를 설명하는 중요한 연산으로, 주로 3차원 공간에서 사용됩니다. 물리학(Physics) 및 공학(Engineering)의 다양한 분야에서 외적의 응용은 필수적입니다.

\subsection{왜 "외적"인가? (Explanation of the Term "Cross Product")}

\noindent 한국어로 "외적"은 두 벡터 간의 곱을 의미하는 수학적 연산을 설명할 때 사용됩니다. "외(外)"는 "바깥" 또는 "외부"를 의미하며, 벡터들이 이루는 평면의 바깥 방향, 즉 평면에 수직한 벡터를 나타냅니다. "적(積)"은 곱셈(Multiplication) 또는 누적을 의미하며, 이는 벡터 사이의 곱을 계산하는 연산을 의미합니다.

\section{정의와 목적}

\subsection{벡터 외적의 수학적 정의 (Mathematical Definition of Cross Product)}

\noindent 벡터 \(\mathbf{a}\)와 \(\mathbf{b}\)의 외적(Cross product)은 두 벡터가 만드는 평행사변형(Parallelogram)의 넓이와, 그 평면에 수직한 벡터를 나타냅니다. 외적의 크기(Magnitude)는 다음과 같이 주어집니다:

\[
  \mathbf{a} \times \mathbf{b} = |\mathbf{a}| |\mathbf{b}| \sin(\theta) \mathbf{n}
\]

\noindent 여기서 \(\theta\)는 두 벡터 사이의 각도(Angle)이고, \(\mathbf{n}\)은 오른손 법칙(Right-hand rule)에 따른 방향의 단위 벡터(Unit vector)입니다.

\vspace{1\baselineskip}
\noindent \emoji{shopping} e.g. 벡터 \(\mathbf{a} = (1, 0, 0)\)와 \(\mathbf{b} = (0, 1, 0)\)의 외적은:
\[
  \mathbf{a} \times \mathbf{b} = |\mathbf{a}| |\mathbf{b}| \sin(90^\circ) \hat{k} = 1 \times 1 \times 1 \hat{k} = \hat{k} = (0, 0, 1)
\]
이는 \(x\)-축과 \(y\)-축을 이루는 평면에 수직한 \(z\)-축 방향의 단위 벡터입니다.

\subsection{성분별 외적 계산 (Component-wise Cross Product Calculation)}

\noindent 외적은 각 성분별로 교차 곱(Cross multiplication)을 통해 계산됩니다. 두 벡터 \(\mathbf{a} = (a_x, a_y, a_z)\)와 \(\mathbf{b} = (b_x, b_y, b_z)\)의 외적은 다음과 같은 행렬식(Determinant)으로 구할 수 있습니다:

\[
  \mathbf{a} \times \mathbf{b} = \begin{vmatrix}
    \hat{i} & \hat{j} & \hat{k} \\
    a_x     & a_y     & a_z     \\
    b_x     & b_y     & b_z
  \end{vmatrix}
\]

\vspace{1\baselineskip}
\noindent \emoji{shopping} e.g. 벡터 \(\mathbf{a} = (2, 3, 4)\)와 \(\mathbf{b} = (5, 6, 7)\)의 외적은:
\[
  \mathbf{a} \times \mathbf{b} = \begin{vmatrix}
    \hat{i} & \hat{j} & \hat{k} \\
    2       & 3       & 4       \\
    5       & 6       & 7
  \end{vmatrix} = \hat{i}(3 \cdot 7 - 4 \cdot 6) - \hat{j}(2 \cdot 7 - 4 \cdot 5) + \hat{k}(2 \cdot 6 - 3 \cdot 5)
\]
\[
  = \hat{i}(-3) - \hat{j}(-6) + \hat{k}(-3) = (-3, 6, -3)
\]

\subsection{외적의 차원성 (Dimensionality of Cross Product)}

\noindent 외적은 \textbf{3차원}(3D)에서만 정의됩니다. 2차원 공간에서는 두 벡터가 만드는 평면에 수직한 벡터를 정의할 수 없기 때문에, 외적이 성립하지 않습니다. 3차원에서 두 벡터가 이루는 평면에 수직한 벡터만이 외적의 결과로 나타날 수 있습니다.

\section{외적의 물리적 및 수학적 의미 (Physical and Mathematical Meaning of Cross Product)}

\subsection{외적의 본질: 기하학적 모델 (Geometrical Model of Cross Product)}

\noindent 외적은 벡터 간의 평면을 정의하고, 이 평면에 수직한 벡터를 생성합니다. 이를 통해 벡터 간의 기하학적 관계를 직관적으로 이해할 수 있습니다.

\subsection{벡터 간의 평면 정의 및 방향 (Plane Definition and Direction between Vectors)}

\noindent 두 벡터가 이루는 평면에 수직한 방향은 오른손 법칙(Right-hand rule)에 따라 결정됩니다. 이 평면에 수직한 벡터는 외적의 결과입니다.

\subsection{오른손 법칙과 물리적 배경 (Right-hand Rule and Physical Background)}

\noindent 오른손 법칙은 두 벡터의 외적에서 결과 벡터의 방향을 결정하는 규칙입니다. 오른손을 사용해 첫 번째 벡터에 손가락을 따라가면, 두 번째 벡터 방향으로 손가락이 굽혀지고, 엄지가 가리키는 방향이 외적의 결과 벡터(Direction of cross product)의 방향입니다.

예를 들어, 전류(Current)가 흐르는 도선 주위에 생성되는 자기장의 방향은 오른손 법칙에 의해 결정됩니다. 전류의 방향을 따라 엄지를 두고, 나머지 손가락이 자기장(Magnetic field)의 방향을 나타냅니다.

\subsection{외적의 결과가 오른손 법칙을 따르는 이유 (Why Cross Product Follows the Right-hand Rule)}

외적(Cross product)은 두 벡터가 정의하는 평면에 수직한 벡터를 생성합니다. 이 벡터의 방향을 결정하는 데 오른손 법칙을 사용합니다. 이 법칙은 벡터의 순서와 방향을 명확히 하기 위한 규칙으로, 다음과 같은 이유로 채택되었습니다:

\begin{itemize}
  \item \textbf{벡터의 순서와 방향 명확화}: 두 벡터 \(\mathbf{a}\)와 \(\mathbf{b}\)의 외적 \(\mathbf{a} \times \mathbf{b}\)의 방향은 벡터 \(\mathbf{a}\)에서 \(\mathbf{b}\)로 이동하는 방향에 수직입니다. 오른손 법칙은 이 방향을 일관되게 결정할 수 있는 방법을 제공합니다. 만약 두 벡터의 순서가 바뀌면 결과 벡터의 방향도 반대가 되기 때문에, 법칙을 사용하여 방향성을 정의합니다.
  \item \textbf{물리적 일관성}: 오른손 법칙은 물리학에서 회전, 자이로스코프, 전자기학 등 다양한 응용에서 일관성을 제공합니다. 예를 들어, 회전 운동의 방향, 자기장과 전류의 상호작용 등에서 이 법칙이 자연스럽게 적용됩니다.
  \item \textbf{3D 벡터 공간의 표준}: 3차원 공간에서는 외적의 방향이 2차원 평면에서 수직을 정의해야 합니다. 오른손 법칙을 사용하면 이 방향을 일관되게 정의할 수 있어, 다양한 수학적 및 물리적 문제를 해결하는 데 유용합니다.
\end{itemize}

\section{결론}

\noindent 벡터 외적은 3차원 공간에서 벡터 간의 기하학적 관계를 명확히 이해하는 데 중요한 도구입니다. 오른손 법칙은 외적의 방향성을 결정하는 직관적이고 유용한 방법입니다.

\end{document}
