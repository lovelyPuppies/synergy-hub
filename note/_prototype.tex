\documentclass[12pt]{article}
\usepackage{emoji}  % Emoji package for rendering emojis
\usepackage{fontspec} % Required for font settings
\usepackage[utf8]{inputenc}

% Set the main font to Noto Sans CJK KR for Korean and English
\setmainfont{Noto Sans CJK KR}

\usepackage{xcolor}   % Required for color settings
% Set background color and text color
\pagecolor{black}  % Set the background color to black
\color{white}      % Set the text color to white

\usepackage{amsmath}  % Required for mathematical symbols
\usepackage{graphicx} % Required for including images

\title{Noto Sans CJK KR와 색상 설정 및 이모지 지원 예시}
\author{}
\date{}

\begin{document}

\maketitle

\section{서론}
이 문서는 \textit{Noto Sans CJK KR} 폰트를 사용하여 작성되었습니다. 배경은 검정색이고, 글자는 흰색으로 설정되었습니다. 아래는 이모지를 사용한 예시입니다.

이모지 예시: \emoji{shopping} \emoji{leaves} \emoji{rose}

\textbf{굵은 글씨 예시}입니다. 렌더링 속도 어떤가

\textit{기울임꼴 예시}입니다. (영어 텍스트에서는 기울임이 가능합니다.)

\section{수학 예시}
다음은 수학 예시입니다:

\[
\hat{\mathbf{v}} = \frac{\mathbf{v}}{|\mathbf{v}|}
\]

\section{이모지 추가 예시}
Here are some more emojis using the \texttt{emoji} package:

\emoji{woman-health-worker-medium-skin-tone} \emoji{family-man-woman-girl-boy}

Flags:

\emoji{flag-malaysia} \emoji{flag-united-kingdom}

\section{한글 예시}
다음은 한글 텍스트 예시입니다:

이 문장은 Noto Sans CJK KR 폰트를 사용하고, 배경색은 검정, 글자색은 흰색으로 설정되었습니다.

\end{document}
