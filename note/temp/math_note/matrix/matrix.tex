\section{행렬}

\subsection{배경과 역사}
행렬(matrix)의 개념은 19세기 초 수학자 아서 케일리(Arthur Cayley)에 의해 처음 도입되었습니다. 케일리는 행렬을 이용하여 선형 변환(linear transformation)을 기하학적으로 해석하였으며, 이는 기하학, 물리학, 경제학 등 여러 분야에서 중요한 역할을 하게 되었습니다. 특히, 선형 방정식을 해결하고, 벡터 공간의 변환을 분석하는 수학적 도구로 자리 잡았습니다.

행렬의 기원은 기하학적 문제를 해결하기 위한 선형 변환에서 출발했지만, 현대에 들어서는 컴퓨터 그래픽스, 머신 러닝(machine learning), 물리학 등에서 필수적인 도구로 발전했습니다. 이로 인해 행렬은 다차원 데이터를 처리하고, 복잡한 시스템을 모델링하는 데 중요한 역할을 하게 되었습니다.

\subsection{정의와 목적}
행렬은 \(m \times n\) 크기의 직사각형 배열로, 각 원소는 특정한 규칙에 따라 나열됩니다. 주로 벡터(vector)의 집합으로 해석되며, 선형 변환을 설명하기 위한 기초적인 도구로 사용됩니다. 행렬 \(A\)는 일반적으로 다음과 같이 나타냅니다:
\[
  A = \begin{pmatrix} a_{11} & a_{12} & \cdots & a_{1n} \\ a_{21} & a_{22} & \cdots & a_{2n} \\ \vdots & \vdots & \ddots & \vdots \\ a_{m1} & a_{m2} & \cdots & a_{mn} \end{pmatrix}
\]
여기서 \(A\)의 각 원소 \(a_{ij}\)는 행렬의 \(i\)-번째 행과 \(j\)-번째 열에 위치합니다.

행렬의 목적은 다차원 데이터를 효율적으로 표현하고, 이를 통해 선형 방정식, 변환 및 데이터를 처리하는 다양한 연산을 수행하는 것입니다.

\subsection{행렬의 성질}
행렬의 다양한 성질들은 행렬이 여러 분야에서 활용될 수 있게 만드는 중요한 요소들입니다. 주요 성질에는 다음이 포함됩니다:

\subsubsection{교환법칙 (Commutative Property)}
행렬 곱셈(matrix multiplication)에서 교환법칙은 성립하지 않습니다. 즉, \( A \times B \neq B \times A \)입니다. 이는 벡터의 방향성 및 차원에 의한 결과로, 두 행렬의 곱셈 순서가 다르면 그 결과도 달라지기 때문입니다.

\vspace{1\baselineskip}
\noindent \emoji{shopping} e.g.
행렬 \( A = \begin{pmatrix} 1 & 2 \\ 3 & 4 \end{pmatrix} \)와 \( B = \begin{pmatrix} 5 & 6 \\ 7 & 8 \end{pmatrix} \)에 대해:
\[
  A \times B = \begin{pmatrix} 1 \times 5 + 2 \times 7 & 1 \times 6 + 2 \times 8 \\ 3 \times 5 + 4 \times 7 & 3 \times 6 + 4 \times 8 \end{pmatrix} = \begin{pmatrix} 19 & 22 \\ 43 & 50 \end{pmatrix}
\]
그러나,
\[
  B \times A = \begin{pmatrix} 5 \times 1 + 6 \times 3 & 5 \times 2 + 6 \times 4 \\ 7 \times 1 + 8 \times 3 & 7 \times 2 + 8 \times 4 \end{pmatrix} = \begin{pmatrix} 23 & 34 \\ 31 & 46 \end{pmatrix}
\]
이처럼 \( A \times B \neq B \times A \)임을 확인할 수 있습니다.

\subsubsection{항등원 (Identity Element)}

\subsubsection{배경과 역사}
항등원(identity matrix 또는 identity element)의 개념은 선형대수학에서 중요한 위치를 차지합니다. 선형 변환을 거치더라도 원래의 벡터 또는 행렬을 변형시키지 않는 성질을 지니며, 이는 19세기부터 사용되어 선형 시스템의 해를 구하는 데 중요한 역할을 해왔습니다.

\subsubsection{정의와 목적}
항등원 \( I \)은 각 주대각선 원소가 1이고 나머지 원소가 모두 0인 정방행렬(square matrix)입니다. 주어진 행렬 \( A \)에 대해 \( A \times I = A \)가 성립하며, 이는 선형 변환에서 항등원이 본래의 변환을 유지하는 역할을 함을 의미합니다.

예를 들어, \(2 \times 2\) 항등원은 다음과 같이 나타낼 수 있습니다:
\[
  I_2 = \begin{pmatrix} 1 & 0 \\ 0 & 1 \end{pmatrix}
\]

\subsubsection{연산 | 메커니즘의 이유}
항등원은 벡터 또는 행렬에 대해 선형 변환을 가할 때, 그 크기나 방향을 변경하지 않는 유일한 행렬입니다. 이러한 성질은 역행렬 계산이나 선형 방정식 풀이에 있어서 중요한 역할을 합니다.

\vspace{1\baselineskip}
\noindent \emoji{shopping} e.g. 행렬 \( A = \begin{pmatrix} 1 & 2 \\ 3 & 4 \end{pmatrix} \)에 항등원을 곱할 때:
\[
  A \times I_2 = \begin{pmatrix} 1 & 2 \\ 3 & 4 \end{pmatrix} \times \begin{pmatrix} 1 & 0 \\ 0 & 1 \end{pmatrix} = \begin{pmatrix} 1 & 2 \\ 3 & 4 \end{pmatrix}
\]
이는 원래의 행렬을 그대로 유지함을 보여줍니다.

\subsubsection{응용 | 실제 사례}
항등원은 역행렬 계산, 선형 시스템의 해 구하기, 선형 변환의 기초 작업에 필수적입니다. 컴퓨터 그래픽스에서 회전, 스케일링 등의 변환을 수행할 때 항등원을 이용하여 초기 상태를 유지하거나 변환 후 결과를 검증하는 데 사용됩니다.

\vspace{1\baselineskip}
\noindent \emoji{shopping} e.g.
컴퓨터 그래픽스에서 객체의 변환이 이루어지기 전에, 객체의 초기 상태를 항등행렬을 사용해 설정한 뒤 이후 변환을 진행하여 최종 위치나 형태를 결정할 수 있습니다.

\subsection{관련 용어}




\subsubsection{가우스-조던 소거법 (Gaussian-Jordan Elimination)}

\paragraph{배경과 역사}
가우스-조던 소거법은 독일의 수학자 카를 프리드리히 가우스(Carl Friedrich Gauss)가 처음 소개한 가우스 소거법의 확장된 형태로, 이후 요한 조던(Johann Georg Jordan)에 의해 완성되었습니다. 이 방법은 선형 방정식의 해를 구하는 간단하고 직관적인 방식으로, 현대 선형대수학에서 중요한 도구로 자리 잡았습니다.

\paragraph{정의와 목적}
가우스-조던 소거법은 행렬을 단위행렬(identity matrix)로 변환하여 선형 방정식의 해를 구하는 방법입니다. 주어진 행렬에 대해 행 연산을 반복하여 상삼각행렬(upper triangular matrix) 또는 단위행렬로 변환함으로써, 방정식의 해를 명확하게 구할 수 있습니다.

\paragraph{연산 | 메커니즘의 이유}
이 방법은 주어진 행렬을 단계적으로 간단한 형태로 바꾸기 때문에, 계산의 복잡성을 줄일 수 있으며, 동시에 역행렬을 구하는 데에도 사용됩니다. 각 행의 연산은 다른 행에 영향을 주지 않도록 구성되어, 연산이 효율적으로 이루어집니다.

\vspace{1\baselineskip}
\noindent \emoji{shopping} e.g. 행렬 \( A = \begin{pmatrix} 1 & 2 \\ 3 & 4 \end{pmatrix} \)를 가우스-조던 소거법을 통해 단위행렬로 변환하는 과정을 단계적으로 설명하겠습니다:

\noindent 1. 첫 번째 행의 첫 번째 원소 \( a_{11} \)이 \(1\)이므로, 이를 기준으로 두 번째 행을 제거하기 위한 연산을 수행합니다:
\[
  \text{Row 2} = \text{Row 2} - 3 \times \text{Row 1}
\]
그 결과는 다음과 같습니다:
\[
  \begin{pmatrix} 1 & 2 \\ 0 & -2 \end{pmatrix}
\]

\noindent 2. 두 번째 행을 -2로 나누어 \( a_{22} \)를 \( 1 \)로 만듭니다:
\[
  \text{Row 2} = \frac{\text{Row 2}}{-2}
\]
그 결과는:
\[
  \begin{pmatrix} 1 & 2 \\ 0 & 1 \end{pmatrix}
\]

\noindent 3. 마지막으로 첫 번째 행의 두 번째 원소를 0으로 만들기 위해, 두 번째 행을 이용하여 첫 번째 행을 연산합니다:
\[
  \text{Row 1} = \text{Row 1} - 2 \times \text{Row 2}
\]
그 결과 단위행렬을 얻습니다:
\[
  \begin{pmatrix} 1 & 0 \\ 0 & 1 \end{pmatrix}
\]

\paragraph{응용 | 실제 사례}
가우스-조던 소거법은 선형 방정식의 해를 구하는 데 널리 사용되며, 컴퓨터 알고리즘에서 효율적으로 구현됩니다. 특히, 대규모 시스템의 연립 방정식을 해결하는 데 매우 유용합니다.

\subsubsection{Matrix Multiplication (행렬 곱셈)}

\paragraph{배경과 역사}
행렬 곱셈(Matrix Multiplication)은 선형 변환의 조합을 나타내기 위해 처음 도입되었으며, 기하학적 변환을 설명하기 위해 19세기 후반에 선형대수학의 중요한 부분으로 발전하였습니다. 현재는 컴퓨터 과학, 물리학, 통계학, 기계 학습 등 다양한 분야에서 데이터를 변환하거나 모델링하는 데 필수적인 연산입니다.

\paragraph{정의와 목적}
두 행렬 \(A\)와 \(B\)의 곱 \(C = A \times B\)는 \(A\)의 각 행 벡터와 \(B\)의 각 열 벡터 간의 내적을 계산한 값으로 정의됩니다. 구체적으로, \(A\)가 \(m \times n\) 행렬이고 \(B\)가 \(n \times p\) 행렬일 때, 결과 행렬 \(C\)는 \(m \times p\) 행렬이 됩니다:
\[
  C_{ij} = \sum_{k=1}^{n} A_{ik} B_{kj}
\]
행렬 곱셈은 선형 변환을 연속적으로 적용할 때 매우 유용한 연산입니다.

\paragraph{연산 | 메커니즘의 이유}
행렬 곱셈은 행렬의 차원과 방향성에 기초하여 정의되며, \(A\)의 열 수와 \(B\)의 행 수가 일치해야만 곱셈이 가능합니다. 이 구조는 선형 변환을 두 번 연속으로 적용하는 것과 동등하며, 복잡한 변환을 하나로 결합할 수 있게 해줍니다.

\vspace{1\baselineskip}
\noindent \emoji{shopping} e.g. 두 행렬 \(A = \begin{pmatrix} 1 & 2 \\ 3 & 4 \end{pmatrix}\)와 \(B = \begin{pmatrix} 5 & 6 \\ 7 & 8 \end{pmatrix}\)에 대해:
\[
  A \times B = \begin{pmatrix} 1 \times 5 + 2 \times 7 & 1 \times 6 + 2 \times 8 \\ 3 \times 5 + 4 \times 7 & 3 \times 6 + 4 \times 8 \end{pmatrix} = \begin{pmatrix} 19 & 22 \\ 43 & 50 \end{pmatrix}
\]

\paragraph{행렬 곱과 벡터 내적의 차이}
벡터 내적은 두 벡터의 방향과 크기에 관한 정보를 주는 스칼라 값을 반환하는 연산입니다. 반면, 행렬 곱은 두 행렬 간의 선형 변환을 결합하여 \textbf{또 다른 행렬}을 반환합니다. 따라서, 행렬 곱은 스칼라 값이 아닌 행렬을 결과로 반환하며, 이는 변환의 결과를 나타냅니다.

\paragraph{응용 | 실제 사례}
행렬 곱셈은 컴퓨터 그래픽스에서 3D 객체를 변환하거나 회전할 때, 기계 학습 모델에서 데이터를 처리할 때, 데이터 분석에서 변수 간의 관계를 계산할 때 널리 사용됩니다. 특히, 신경망에서 가중치 행렬과 입력 데이터를 곱하여 결과를 계산하는 데 사용됩니다.

\vspace{1\baselineskip}
\noindent \emoji{shopping} e.g.
컴퓨터 그래픽스에서 객체를 이동시키는 변환 행렬 \(T\)와 회전 행렬 \(R\)을 곱하여 두 변환을 동시에 적용할 수 있습니다:
\[
  T = \begin{pmatrix} 1 & 0 & 5 \\ 0 & 1 & 3 \\ 0 & 0 & 1 \end{pmatrix}, \quad R = \begin{pmatrix} \cos{\theta} & -\sin{\theta} & 0 \\ \sin{\theta} & \cos{\theta} & 0 \\ 0 & 0 & 1 \end{pmatrix}
\]
결과적으로 \(T \times R\)은 객체를 이동시키고 회전시키는 복합 변환을 나타냅니다.


\subsubsection{아다마르 곱 (Hadamard Product)}

\paragraph{배경과 역사}
아다마르 곱(Hadamard Product)은 수학자 자크 아다마르(Jacques Hadamard)가 처음 정의한 성분별 곱셈 연산입니다. 행렬 곱과는 다르게, 아다마르 곱은 두 행렬의 동일한 위치에 있는 성분들끼리 곱하는 연산입니다. 이는 선형 변환보다는 주로 성분별 연산에 사용되며, 기계 학습과 데이터 분석에서 널리 응용됩니다.

\paragraph{정의와 목적}
아다마르 곱은 두 행렬 \( A \)와 \( B \)의 성분별 곱을 계산하는 연산으로, \( A \circ B \)로 표현됩니다. 두 행렬이 \( m \times n \)일 때, 아다마르 곱 \( C = A \circ B \)의 성분 \( C_{ij} \)는 다음과 같이 정의됩니다:
\[
  C_{ij} = A_{ij} \times B_{ij}
\]
이는 행렬 곱과는 달리 성분별로 독립적으로 곱해지는 연산으로, 차원이 같은 두 행렬에 대해서만 수행됩니다.

\paragraph{연산 | 메커니즘의 이유}
아다마르 곱은 성분별로 독립적으로 수행되며, 선형 변환을 나타내지 않습니다. 대신, 각 성분의 값을 독립적으로 곱하여 계산할 수 있으며, 이는 각 데이터 포인트에 대한 독립적인 처리가 필요할 때 유용합니다.

\vspace{1\baselineskip}
\noindent \emoji{shopping} e.g. 두 행렬 \( A = \begin{pmatrix} 1 & 2 \\ 3 & 4 \end{pmatrix} \)와 \( B = \begin{pmatrix} 5 & 6 \\ 7 & 8 \end{pmatrix} \)의 아다마르 곱은 다음과 같이 계산됩니다:
\[
  A \circ B = \begin{pmatrix} 1 \times 5 & 2 \times 6 \\ 3 \times 7 & 4 \times 8 \end{pmatrix} = \begin{pmatrix} 5 & 12 \\ 21 & 32 \end{pmatrix}
\]

\paragraph{응용 | 실제 사례}
아다마르 곱은 기계 학습에서 매우 자주 사용되는 연산입니다. 특히, 신경망에서 가중치 업데이트나 성분별 처리에 사용되며, 이미지 처리, 필터링 등의 분야에서도 활용됩니다. 또한, 데이터 분석에서는 성분별로 가중치를 적용하여 변수 간의 독립적인 영향을 분석할 때 유용하게 사용됩니다.

\vspace{1\baselineskip}
\noindent \emoji{shopping} e.g. 신경망에서 출력 값과 가중치 행렬의 성분별 곱을 구할 때, 아다마르 곱을 사용하여 각 노드 간의 독립적인 연산을 처리할 수 있습니다.





\subsubsection{행렬의 대각화 (Matrix Diagonalization)}

\paragraph{배경과 역사}
행렬의 대각화(Matrix Diagonalization)는 선형대수학에서 중요한 개념으로, 19세기 후반에 행렬 이론이 발전하면서 등장했습니다. 대각화는 주로 복잡한 행렬을 대각선 형태로 변환하여 연산을 단순화하고, 이를 통해 선형 변환을 더 쉽게 분석할 수 있습니다. 이 개념은 물리학에서 양자역학, 컴퓨터 그래픽스에서 3D 변환, 그리고 데이터 분석에서 차원 축소 등 다양한 분야에서 활용됩니다.

\noindent 대각화는 대칭 행렬이나 특정 조건을 만족하는 행렬에서만 가능하며, 이러한 특성은 스펙트럼 이론(spectral theory)과 같은 선형대수학의 중요한 주제를 형성합니다.

\paragraph{정의와 목적}
행렬의 대각화는 주어진 정방행렬 \( A \)를 대각 행렬 \( D \)와 그에 상응하는 행렬 \( P \)로 분해하는 과정입니다. 구체적으로, \( A = P D P^{-1} \)라는 관계가 성립하며, 여기서 \( D \)는 대각행렬, \( P \)는 고유벡터(eigenvector)들로 이루어진 행렬입니다. 대각화가 가능한 경우, 복잡한 연산을 대각행렬을 통해 더 간단하게 처리할 수 있습니다.

\noindent 대각화의 주요 목적은 고유값과 고유벡터를 이용하여 행렬을 변환하고, 이를 통해 연산을 단순화하는 데 있습니다.

\paragraph{연산 | 메커니즘의 이유}
대각화는 행렬이 고유값(eigenvalue)과 고유벡터(eigenvector)를 가질 때, 이 값들을 이용하여 원래의 행렬을 더 단순한 형태로 바꾸는 과정입니다. 이때, \( A \)의 고유값들이 대각행렬 \( D \)의 대각 원소로 나타나며, \( P \)는 이 고유값에 대응하는 고유벡터들로 구성됩니다.

\noindent 대각화의 중요한 성질은 복잡한 행렬 연산을 대각선 형태의 행렬로 변환하여 쉽게 계산할 수 있다는 점입니다. 예를 들어, 행렬의 거듭제곱 연산에서 대각화된 행렬을 사용하면 연산이 매우 간단해집니다:
\[
  A^k = P D^k P^{-1}
\]
여기서 \( D^k \)는 대각행렬의 대각 원소들을 각각 \(k\)제곱한 형태입니다.

\vspace{1\baselineskip}
\noindent \emoji{shopping} e.g. 예를 들어, 행렬 \( A = \begin{pmatrix} 4 & 1 \\ 1 & 4 \end{pmatrix} \)의 고유값은 \( \lambda_1 = 3 \), \( \lambda_2 = 5 \)이고, 대응하는 고유벡터는 \( \mathbf{v_1} = \begin{pmatrix} 1 \\ -1 \end{pmatrix} \), \( \mathbf{v_2} = \begin{pmatrix} 1 \\ 1 \end{pmatrix} \)입니다. 이때 행렬 \( A \)는 대각화될 수 있으며:
\[
  A = P D P^{-1}, \quad \text{where} \quad D = \begin{pmatrix} 3 & 0 \\ 0 & 5 \end{pmatrix}, \quad P = \begin{pmatrix} 1 & 1 \\ -1 & 1 \end{pmatrix}
\]
따라서, \( A^k \)는 \( P D^k P^{-1} \)로 계산할 수 있습니다.

\paragraph{응용 | 실제 사례}
행렬의 대각화는 다양한 실세계 응용에서 중요한 역할을 합니다. 예를 들어, 양자역학에서 해밀토니언(Hamiltonian) 연산자는 대각화 과정을 통해 에너지 고유상태로 분해될 수 있으며, 이를 통해 복잡한 물리 시스템을 분석할 수 있습니다.

\noindent 또한, 컴퓨터 그래픽스에서는 회전, 스케일링과 같은 선형 변환을 대각화하여 처리할 수 있으며, 기계 학습에서는 데이터 분석에서 차원 축소(dimensionality reduction)와 특성 분해에 사용됩니다.




\subsubsection{Determinant (판별식)}

\paragraph{배경과 역사}
Determinant(판별식)의 개념은 17세기 수학자 라이프니츠(Gottfried Wilhelm Leibniz)에 의해 처음 도입되었습니다. 선형 방정식의 해 존재 여부를 결정하는 데 중요한 역할을 했으며, 시간이 지나면서 선형 변환에서 행렬이 공간을 어떻게 변형시키는지 측정하는 도구로 발전했습니다.

\paragraph{정의와 목적}
Determinant는 주어진 행렬이 공간을 변환하는 정도를 나타내는 값입니다. \(n \times n\) 정방행렬 \(A\)의 determinant는 다음과 같이 정의됩니다:
\[
  \det(A) = \sum_{\sigma \in S_n} \text{sgn}(\sigma) \prod_{i=1}^n a_{i, \sigma(i)}
\]
여기서 \(S_n\)은 \(n\)-차원 순열(permutation)들의 집합이고, \(\text{sgn}(\sigma)\)은 순열의 부호(sign)를 의미합니다. 이 수식은 모든 가능한 순열을 고려하여 행렬의 특정한 구조를 분석합니다.

\paragraph{연산 | 메커니즘의 이유}
Determinant 수식에서 핵심 개념은 \textbf{순열(permutation)*}입니다. 주어진 \(n \times n\) 행렬에서 각 행과 열을 어떻게 섞을 수 있는지를 나타내며, 순열의 부호는 각 순열이 원래 배열을 어떻게 변형시키는지를 보여줍니다. 각 순열에 대한 부호가 붙은 곱셈의 합을 통해 전체 행렬의 크기와 방향 변화를 측정할 수 있습니다.

\vspace{1\baselineskip}
\noindent \emoji{shopping} e.g.
\(2 \times 2\) 행렬 \(A = \begin{pmatrix} 1 & 2 \\ 3 & 4 \end{pmatrix}\)의 경우 determinant는 다음과 같이 계산됩니다:
\[
  \det(A) = (1 \times 4) - (2 \times 3) = -2
\]
여기서 \( (1 \times 4) \)는 대각선 원소의 곱이고, \( (2 \times 3) \)는 비대각선 원소의 곱으로서, 전체 determinant를 계산하는 방식입니다.

\paragraph{응용 | 실제 사례}
Determinant는 컴퓨터 그래픽스에서 이미지의 회전 및 스케일링을 처리하는 데 사용되며, 물리학에서는 좌표 변환을 통해 물체의 움직임을 분석하는 데 활용됩니다. 또한, 기계 학습에서는 데이터셋의 특성 공간에서 변수들 간의 상관 관계를 분석할 때 determinant를 사용합니다.
